\documentclass[12pt, letterpaper]{article}
\usepackage[colorlinks=true,linkcolor=black,citecolor=blue,filecolor=cyan,pagecolor=blue]{hyperref} 
\usepackage[toc,style=altlistgroup,hyperfirst=false]{glossaries}
\usepackage[utf8]{inputenc} %para poder escribir símbolos no anglosajones 
\usepackage[spanish, mexico]{babel} %Escribir en español (acentos)
\usepackage[T1]{fontenc}
\usepackage{amssymb}
\usepackage{mathtools}
\usepackage[usenames]{color}
\usepackage{float}
\usepackage{graphicx}  %%para las imagenes
\usepackage{cite} % para contraer referencias
\usepackage{multicol}
\usepackage{multirow}
\usepackage{bm}
\usepackage{bbm}
\usepackage[left=2.5cm,top=2.5cm,right=2.5cm,bottom=2.5cm]{geometry}
\parindent=5mm
\graphicspath{{images/}}
\usepackage{etoolbox}
\let\bbordermatrix\bordermatrix
\patchcmd{\bbordermatrix}{8.75}{4.75}{}{}
\patchcmd{\bbordermatrix}{\left(}{\left[}{}{}
\patchcmd{\bbordermatrix}{\right)}{\right]}{}{}
%%%%glosario
\makeindex
%\makeglossaries
%\input{./glosario.tex}

%%%%%%%%%%%%%%%%%%%%%%%%%%%%%%%%%%%%%%%%%%%%%%%%%%%%%%%%%%%%%%%%%%%%%%%%%%%%%
%%NOTA IMPORTANTE:
%%Para relacionar el glosario.tex con este archivo
%%Es necesario abrir la terminal (Simbolo del sistema en windows)
%%Ir a la carpeta contenedora y escribir el siguiente comando:
%%makeindex -s PROYECTO_final.ist -t PROYECTO_final.glg -o PROYECTO_final.gls PROYECTO_final.glo
%%%%%%%%%%%%%%%%%%%%%%%%%%%%%%%%%%%%%%%%%%%%%%%%%%%%%%%%%%%%%%%%%%%%%%%%%%%%%

%%%% inicio del documento
\begin{document}

\thispagestyle{empty}

%%%%%%% portada

\thispagestyle{empty}

\begin{minipage}[c][0.1\textheight][c]{0.2\textwidth}
\begin{center}
    \includegraphics[width=4cm, height=4cm]{cimat}
\end{center}
\end{minipage}
\begin{minipage}[c][0.1\textheight][t]{0.8\textwidth}
\begin{center}
    {\hspace{2cm}\scshape Centro de Investigación en Matemáticas}
    \vspace{-.5cm}
\end{center}
\hspace*{1.0cm} \rule[0mm]{0.9\textwidth}{0.8mm}
\hspace*{1.17cm}   \rule[4mm]{0.9\textwidth}{0.1mm}
    \vspace{-1cm}
\begin{center}
    { \hspace{2cm}\scshape  Unidad Monterrey}
\end{center}
\end{minipage}

\begin{minipage}[c][0.6\textheight][t]{0.2\textwidth}
\begin{center}
\hskip2pt
\vrule width2.5pt height10cm
        \hskip1mm
        \vrule width1pt height10cm \\ \vspace{2cm}
        \includegraphics[height=4.5cm]{mty}
        \end{center}
\end{minipage}
\begin{minipage}[c][0.9\textheight][t]{0.65\textwidth}
  \begin{center}

	
    \vspace{3.2cm}
    
%%%% TITULO EN PORTADA

  \scshape Proyecto No. 1.\\ \normalsize
  
  \vspace{2cm}  
  
    
            
    Métodos multivariados de Análisis de Datos\\
    \vspace{1cm}   
    Análisis de nutrientes en pizzas.\\
    \vspace{1cm}   
    \vspace{1cm}   
    Ricardo Cruz Sánchez\\
    Rolando Corona Jiménez
    \vspace{.5cm}   
  \end{center}
  
\end{minipage}

%TABLA DE INDICES
\pagebreak
\tableofcontents

\cleardoublepage
%INTRODUCCIÓN
\pagebreak
\section{Introducción.}
La bromatología es la ciencia encargada del análisis de los nutrientes contenidos en los alimentos. Los estudios relacionados con esta disciplina cobran importancia al considerar que existen cantidades recomendadas en la ingesta diaria de cualquier individuo y el incremento o decremento de las cantidades repercute directamente en la salud del la persona.\\

Las porciones de nutrientes que posee cierto alimento en particular, se determinan a través de diversas pruebas de laboratorio, las cuales buscan ser lo más precisas y por lo general reportan los nutrientes contenidos en 100 gramos del alimento en cuestión.\\

Los alimentos considerados como \emph{comida rápida} suelen tener cantidades elevadas en los nutrientes, por lo que la ingesta de este tipo de alimentos suele sobrepasar o aportar considerablemente a los límites recomendados.\\

Particularmente, la pizza, tiende a ser uno de los alimentos con los cuales se sobrepasan los límites de nutrientes recomendados. Esto, principalmente, se debe a que es una mezcla de ingredientes cuya aportación nutrimental es elevada, a saber, harina, tomate, queso y carne.\\
 
En el presente trabajo, se considera una base de datos relativa a las pruebas nutrimentales realizadas en distintas pizzas y con base a estos datos se pretende realizar un análisis multivariado.\\

El documento se divide en 4 secciones, la primera de ellas realiza un análisis exploratorio de los datos, donde se presentan las variables y resumen de datos a través medidas en forma de gráficas. Esta exploración se hace con el fin de conocer mejor las variables y tener una idea a priori de los resultados esperados. Posteriormete, la segunda sección implementa los modelos de reducción de dimensiones, enfocados en trabajar en un espacio que permita una mejor visualización de los datos. La siguiente sección, presenta los modelos de clasificación, con los que se espera asignar una categoría a cada tupla de datos, considerados como variables predictivas. Por último, la última sección resume los resultados obtenidos y plantea las mejoras que se pueden realizar a este tipo de trabajos.


\pagebreak

\section{Análisis exploratorio.}

\section{Modelos de reducción de dimensiones.}

\section{Modelos de clasificación.}

\section{Conclusiones.}



\end{document}