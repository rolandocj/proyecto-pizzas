\documentclass[pdf]{beamer}
%\documentclass[notes]{beamer}
%\documentclass{beamer}
\usepackage[utf8]{inputenc}
\usepackage{lmodern}
\usepackage{colortbl}
\usepackage{adjustbox}
%\usepackage{scrextend}
%\changefontsizes{7.5pt}


\makeatletter
\makeatother
\usepackage{graphicx}

\mode<presentation>{\usetheme{Warsaw}}
%\mode<presentation>{\usetheme{Madrid}}
%% preamble
\title{Métodos multivariados de Análisis de Datos}
\subtitle{Análisis de nutrientes en pizzas}

\author{
Ricardo Cruz Sánchez \\
  \and
Rolando Corona Jiménez
}

\institute[CIMAT]{CIMAT}

\AtBeginSection[]{%
\begin{frame}
    \tableofcontents[currentsection, subsectionstyle=show/show/hide]
\end{frame}
}

\begin{document}

\begin{frame}
\titlepage
\end{frame}

%\AtBeginSubsection[]
%{
%  \begin{frame}<beamer>
%    \frametitle{Contenido}
%    \tableofcontents[currentsection,currentsubsection]
%  \end{frame}
%}



\section{Introducción}
\begin{frame}{Proposito y objetivos}
\begin{block}{Objetivo general}
Generar un modelo confiable con el cual se pueda detectar de manera eficiente, econ\'omica y sencilla la presencia de DM2 en la población mexicana femenina.
\end{block}

\begin{block}{objetivo secundario}
\begin{enumerate}
	\item Incorporar a la poblaci\'on masculina en el estudio.
	\item Crear política públicas encaminadas a erradicar el padecimiento con base en los resultados generados por el modelo.
	\item Disminuir los costos que se generan en los centros de salud causados por la detecci\'on tard\'ia de DM2
\end{enumerate}
\end{block}
\end{frame}

\section{Análisis exploratorio.}

\begin{frame}
\begin{itemize}
\item \textbf{Ident:} Variable tipo numérica, la cual corresponde a un identificador para cada pizza.
\item \textbf{HUMED:} Variable tipo numérica que indica el porcentaje de humedad contenido en la pizza.
\item \textbf{PROTE:} Variable tipo numérica que indica la cantidad de gramos de proteina contenida en 100g de pizza.
\item \textbf{GRASA:} Variable tipo numérica que indica la cantidad de gramos de grasa contenida en 100g de pizza.
\item \textbf{CENIZA:} Variable tipo numérica que indica la cantidad de gramos de ceniza contenida en 100g de pizza.
\item \textbf{SODIO:} Variable tipo numérica que indica la cantidad de gramos de sodio contenida en 100g de pizza.
\item \textbf{CARBO:} Variable tipo numérica que indica la cantidad de gramos de carbohidratos contenida en 100g de pizza.
\item \textbf{CALOR:} Variable tipo numérica que indica la cantidad de calorias contenida en 100g de pizza.
\item \textbf{MARCA:} Varibale tipo string, la cual indica la marca que fabrico la pizza. Es una variable categórica.
\end{itemize}
\end{frame}

\begin{frame}
\begin{figure}[h]
\centering
\includegraphics[scale=1]{images/marca.png} 
\label{i1}
\caption{Conteo de registros por marca}
\end{figure}
\end{frame}

\begin{frame}
\begin{figure}[h]
\centering
\includegraphics[scale=1]{images/ident.png} 
\label{i2}
\caption{Comportamiento de la variable Ident de acuerdo a la marca}
\end{figure}
\end{frame}


\begin{frame}
\begin{figure}[h]
\centering
\includegraphics[scale=1]{images/humed.png} 
\label{i3}
\caption{Comportamiento de la variable Humed de acuerdo a la marca}
\end{figure}
\end{frame}


\begin{frame}
\begin{figure}[h]
\centering
\includegraphics[scale=1]{images/prote.png} 
\label{i4}
\caption{Comportamiento de la variable Prote de acuerdo a la marca}
\end{figure}
\end{frame}


\begin{frame}
\begin{figure}[h]
\centering
\includegraphics[scale=1]{images/grasa.png} 
\label{i5}
\caption{Comportamiento de la variable Grasa de acuerdo a la marca}
\end{figure}
\end{frame}


\begin{frame}
\begin{figure}[h]
\centering
\includegraphics[scale=1]{images/ceniz.png} 
\label{i6}
\caption{Comportamiento de la variable CENIZ de acuerdo a la marca}
\end{figure}
\end{frame}


\begin{frame}
\begin{figure}[h]
\centering
\includegraphics[scale=1]{images/sodio.png} 
\label{i7}
\caption{Comportamiento de la variable sodio de acuerdo a la marca}
\end{figure}
\end{frame}

\begin{frame}
\begin{figure}[h]
\centering
\includegraphics[scale=1]{images/carbo.png} 
\label{i8}
\caption{Comportamiento de la variable Carbo de acuerdo a la marca}
\end{figure}
\end{frame}


\begin{frame}
\begin{figure}[h]
\centering
\includegraphics[scale=1]{images/calor.png} 
\label{i9}
\caption{Comportamiento de la variable Calor de acuerdo a la marca}
\end{figure}
\end{frame}


\begin{frame}
\begin{figure}[h]
\centering
\includegraphics[scale=1]{images/pgc.png} 
\label{i10}
\caption{Grasa+Proteina+Carbohidratos para cada observación}
\end{figure}
\end{frame}


\begin{frame}
\begin{figure}[h]
\centering
\includegraphics[scale=1]{images/corr.png} 
\label{i11}
\caption{Correlación de las variables.}
\end{figure}
\end{frame}

\begin{frame}

\end{frame}


\section{Modelos de reducción de dimensiones.}

\subsection{Análisis de componentes principales (PCA)}

\begin{frame}{Cargas de componentes principales}
\begin{table}[ht]
\centering
\begin{tabular}{rrr}
  \hline
 & PC1 & PC2 \\ 
  \hline
Humedad & 0.21 & 0.58 \\ 
  Proteina & -0.47 & -0.03 \\ 
  Grasa & -0.19 & 0.41 \\ 
  Ceniza & -0.51 & 0.15 \\ 
  Sodio & -0.47 & -0.02 \\ 
  Carbohidratos & 0.32 & -0.49 \\ 
  Calorias & -0.34 & -0.48 \\ 
  Varianza acumulada & 48.64 \% &  83.59 \% \\ 
\end{tabular}
	\label{tabla:pesos_PCA}
	\caption{Pesos asociados a las primeras dos componentes principales.}
\end{table}
\end{frame}


\begin{frame}
\begin{figure}[h]
\centering
	\includegraphics[scale=.35]{images/varPCA.png} 
	\label{i_var_PCA}
	\caption{Varianza explicada por las componentes principales}
\end{figure}
\end{frame}


\begin{frame}
\begin{figure}[h]
\centering
	\includegraphics[scale=.35]{images/biplotPCA.png} 
	\label{i_biplot_PCA}
	\caption{Biplot PCA}
\end{figure}
\end{frame}

\subsection{Análisis Factorial}

\begin{frame}{Resumen de Análisis Factorial}
\begin{table}[ht]
\begin{adjustbox}{width= 4in,center}
\centering
\begin{tabular}{rrrrr}
  \hline
 & Factor1 & Factor2 & Varianza específica & Comunalidades \\ 
  \hline
Humedad & 0.06 & -1.00 & 0.01 & 1.00 \\ 
  Proteina & 0.76 & 0.44 & 0.23 & 0.77 \\ 
  Grasa & 0.47 & -0.30 & 0.69 & 0.31 \\ 
  Ceniza & 0.94 & 0.19 & 0.08 & 0.92 \\ 
  Sodio & 0.73 & 0.39 & 0.32 & 0.68 \\ 
  Carbohidratos & -0.90 & 0.44 & 0.01 & 1.00 \\ 
  Calorias & 0.23 & 0.97 & 0.01 & 0.99 \\     
  Prop. de Varianza  & 44 \% &  36.9 \% \\
  Var. acumulada & 44 \% &  80.9 \% \\ 
   \hline
\end{tabular}
\end{adjustbox}
	\label{tabla:factores}
	\caption{Resultados del análisis factorial.}	
\end{table}
\end{frame}

\begin{frame}{Aproximación a R}
\begin{table}[ht]
\begin{adjustbox}{width= 4in,center}
\centering
\begin{tabular}{rrrrrrrr}
  \hline
 & Humedad & Proteina & Grasa & Ceniza & Sodio & Carbohidratos & Calorias \\ 
  \hline
Humedad & -0.00 & -0.02 & 0.00 & -0.00 & 0.01 & -0.00 & -0.00 \\ 
  Proteina & -0.02 & 0.00 & -0.01 & -0.01 & -0.22 & -0.01 & -0.04 \\ 
  Grasa & 0.00 & -0.01 & 0.00 & -0.01 & -0.00 & -0.00 & 0.00 \\ 
  Ceniza & -0.00 & -0.01 & -0.01 & -0.00 & 0.06 & 0.00 & -0.00 \\ 
  Sodio & 0.01 & -0.22 & -0.00 & 0.06 & -0.00 & 0.01 & 0.04 \\ 
  Carbohidratos & -0.00 & -0.01 & -0.00 & 0.00 & 0.01 & -0.00 & -0.00 \\ 
  Calorias & -0.00 & -0.04 & 0.00 & -0.00 & 0.04 & -0.00 & 0.00 \\ 
   \hline
\end{tabular}
\end{adjustbox}
	\label{tabla:aproximacion}
	\caption{Diferencia entre R y $LL' + \Psi$, con redondeo a tres dígitos.}
\end{table}
\end{frame}

\begin{frame}
\begin{figure}[h]
\centering
	\includegraphics[scale=.35]{images/biplotFactores.png} 
	\label{i_biplot_Factores}
	\caption{Biplot usando 2 factores}
\end{figure}
\end{frame}

\section{Análisis por agrupación.}

\subsection{Elección del número de clusters}

\begin{frame}
\begin{figure}[h]
\centering
	\includegraphics[scale=.5]{images/clusterElbow.png} 
	\label{i_cluster_Elbow}
	\caption{Suma total de cuadrados entre clústers vs número de clusters}
\end{figure}
\end{frame}

\begin{frame}
\begin{figure}[h]
\centering
	\includegraphics[scale=.5]{images/clusterSilhouette.png} 
	\label{i_cluster_Silhouette}
	\caption{Silhouette promedio vs número de clusters}
\end{figure}
\end{frame}


\subsection{Asignación de clusters}

\begin{frame}
\begin{figure}[h]
\centering
	\includegraphics[scale=.35]{images/clusterPCA.png} 
	\label{i_cluster_PCA}
	\caption{Clustering jerárquico completo (Representación: PCA)}
\end{figure}
\end{frame}

\begin{frame}
\begin{figure}[h]
\centering
	\includegraphics[scale=.35]{images/clusterFactores.png} 
	\label{i_cluster_Factores}
	\caption{Clustering jerárquico completo (Representación: Factores)}
\end{figure}
\end{frame}


\section{Modelos de clasificación.}
\subsection{LDA}
\begin{frame}

\begin{table}[ht]
\begin{adjustbox}{width= 3.5in,center}
\centering
\begin{tabular}{rrrrrrrrrrrrr}
  \hline
 & A & B & C & D & E & F & G & H & I & J & K & L \\ 
  \hline
A &   9 &   0 &   0 &   0 &   0 &   0 &   0 &   0 &   0 &   0 &   0 &   0 \\ 
  B &   0 &  17 &   0 &   0 &   0 &   0 &   0 &   0 &   0 &   0 &   0 &   0 \\ 
  C &   0 &   0 &  15 &   0 &   0 &   0 &   0 &   0 &   0 &   0 &   0 &   0 \\ 
  D &   0 &   0 &   1 &  19 &   0 &   0 &   0 &   0 &   0 &   0 &   0 &   0 \\ 
  E &   0 &   0 &   0 &   0 &  14 &   2 &   0 &   0 &   0 &   0 &   0 &   0 \\ 
  F &   0 &   0 &   0 &   0 &   1 &  16 &   0 &   0 &   0 &   0 &   0 &   0 \\ 
  G &   0 &   0 &   0 &   0 &   0 &   0 &  12 &   0 &   0 &   0 &   0 &   0 \\ 
  H &   0 &   0 &   0 &   0 &   0 &   0 &   0 &   8 &   9 &   0 &   0 &   0 \\ 
  I &   0 &   0 &   0 &   0 &   0 &   0 &   0 &   2 &  14 &   1 &   0 &   0 \\ 
  J &   0 &   0 &   0 &   0 &   0 &   0 &   0 &   0 &   0 &  13 &   0 &   0 \\ 
  K &   0 &   0 &   0 &   0 &   0 &   0 &   0 &   0 &   0 &   0 &  18 &   0 \\ 
  L &   0 &   0 &   0 &   0 &   0 &   0 &   0 &   0 &   0 &   0 &   0 &  18 \\ 
   \hline
\end{tabular}
\end{adjustbox}
	\label{tabla:confusionLDAtrain}
	\caption{Matriz de confusión LDA train}
\end{table}

error de clasificación:  0.08465608
\end{frame}

\begin{frame}
 \begin{table}[ht]
 \begin{adjustbox}{width= 3.5in,center}
\centering
\begin{tabular}{rrrrrrrrrrrrr}
  \hline
 & A & B & C & D & E & F & G & H & I & J & K & L \\ 
  \hline
A &  21 &   0 &   0 &   0 &   0 &   0 &   0 &   0 &   0 &   0 &   0 &   0 \\ 
  B &   0 &  14 &   0 &   0 &   0 &   0 &   0 &   0 &   0 &   0 &   0 &   0 \\ 
  C &   0 &   0 &  15 &   0 &   0 &   0 &   0 &   0 &   0 &   0 &   0 &   0 \\ 
  D &   0 &   0 &   1 &  12 &   0 &   0 &   0 &   0 &   0 &   0 &   0 &   0 \\ 
  E &   0 &   0 &   0 &   0 &  14 &   1 &   0 &   0 &   0 &   0 &   0 &   0 \\ 
  F &   0 &   0 &   0 &   0 &   0 &  19 &   0 &   0 &   0 &   0 &   0 &   0 \\ 
  G &   0 &   0 &   0 &   0 &   0 &   0 &  19 &   0 &   0 &   0 &   0 &   0 \\ 
  H &   0 &   0 &   0 &   0 &   0 &   0 &   0 &  11 &   4 &   0 &   0 &   0 \\ 
  I &   0 &   0 &   0 &   0 &   0 &   0 &   0 &   0 &  13 &   0 &   0 &   0 \\ 
  J &   0 &   0 &   0 &   0 &   0 &   0 &   0 &   0 &   0 &  20 &   0 &   0 \\ 
  K &   0 &   0 &   0 &   0 &   0 &   0 &   0 &   0 &   0 &   0 &  12 &   0 \\ 
  L &   0 &   0 &   0 &   0 &   0 &   0 &   0 &   0 &   0 &   0 &   0 &  14 \\ 
   \hline
\end{tabular}
\end{adjustbox}
	\label{tabla:confusionLDAtest}
	\caption{Matriz de confusión LDA test}
\end{table}
error de clasificación: 0.03157895
\end{frame}

\subsection{Multinomial}


\begin{frame}
\begin{table}[ht]
\begin{adjustbox}{width= 3.5in,center}
\centering
\begin{tabular}{rrrrrrrrrrrrr}
  \hline
 & A & B & C & D & E & F & G & H & I & J & K & L \\ 
  \hline
A &   9 &   0 &   0 &   0 &   0 &   0 &   0 &   0 &   0 &   0 &   0 &   0 \\ 
  B &   0 &  17 &   0 &   0 &   0 &   0 &   0 &   0 &   0 &   0 &   0 &   0 \\ 
  C &   0 &   0 &  15 &   0 &   0 &   0 &   0 &   0 &   0 &   0 &   0 &   0 \\ 
  D &   0 &   0 &   0 &  20 &   0 &   0 &   0 &   0 &   0 &   0 &   0 &   0 \\ 
  E &   0 &   0 &   0 &   0 &  16 &   0 &   0 &   0 &   0 &   0 &   0 &   0 \\ 
  F &   0 &   0 &   0 &   0 &   0 &  17 &   0 &   0 &   0 &   0 &   0 &   0 \\ 
  G &   0 &   0 &   0 &   0 &   0 &   0 &  12 &   0 &   0 &   0 &   0 &   0 \\ 
  H &   0 &   0 &   0 &   0 &   0 &   0 &   0 &  17 &   0 &   0 &   0 &   0 \\ 
  I &   0 &   0 &   0 &   0 &   0 &   0 &   0 &   0 &  17 &   0 &   0 &   0 \\ 
  J &   0 &   0 &   0 &   0 &   0 &   0 &   0 &   0 &   0 &  13 &   0 &   0 \\ 
  K &   0 &   0 &   0 &   0 &   0 &   0 &   0 &   0 &   0 &   0 &  18 &   0 \\ 
  L &   0 &   0 &   0 &   0 &   0 &   0 &   0 &   0 &   0 &   0 &   0 &  18 \\ 
   \hline
\end{tabular}
\end{adjustbox}
	\label{tabla:confusionMLtrain}
	\caption{Matriz de confusión Multinomial train}
\end{table}

error de clasificación: 0
\end{frame}

\begin{frame}
\begin{table}[ht]
\begin{adjustbox}{width= 3.5in,center}
\centering
\begin{tabular}{rrrrrrrrrrrrr}
  \hline
 & A & B & C & D & E & F & G & H & I & J & K & L \\ 
  \hline
A &  21 &   0 &   0 &   0 &   0 &   0 &   0 &   0 &   0 &   0 &   0 &   0 \\ 
  B &   0 &  14 &   0 &   0 &   0 &   0 &   0 &   0 &   0 &   0 &   0 &   0 \\ 
  C &   0 &   0 &  15 &   0 &   0 &   0 &   0 &   0 &   0 &   0 &   0 &   0 \\ 
  D &   0 &   0 &   1 &  12 &   0 &   0 &   0 &   0 &   0 &   0 &   0 &   0 \\ 
  E &   0 &   0 &   0 &   0 &  15 &   0 &   0 &   0 &   0 &   0 &   0 &   0 \\ 
  F &   0 &   0 &   0 &   0 &   1 &  18 &   0 &   0 &   0 &   0 &   0 &   0 \\ 
  G &   0 &   0 &   0 &   0 &   0 &   0 &  19 &   0 &   0 &   0 &   0 &   0 \\ 
  H &   0 &   0 &   0 &   0 &   0 &   0 &   0 &   9 &   6 &   0 &   0 &   0 \\ 
  I &   0 &   0 &   0 &   0 &   0 &   0 &   0 &   0 &  13 &   0 &   0 &   0 \\ 
  J &   0 &   4 &   0 &   0 &   0 &   2 &   0 &   0 &   0 &  14 &   0 &   0 \\ 
  K &   0 &   0 &   0 &   0 &   0 &   0 &   0 &   0 &   0 &   0 &  12 &   0 \\ 
  L &   0 &   0 &   0 &   0 &   0 &   0 &   0 &   0 &   0 &   0 &   0 &  14 \\ 
   \hline
\end{tabular}
\end{adjustbox}
	\label{tabla:confusionMLtest}
	\caption{Matriz de confusión Multinomial test}
\end{table}

error de clasificación: 0.07368421
\end{frame}

\subsection{Validación cruzada con caret}

\begin{frame}
\begin{table}[ht]
\centering
\begin{tabular}{rrrrrrrrrrrrr}
  \hline
Clasificador & Accuracy\\ 
  \hline
rn &   0.2953684  \\ 
mr &   0.9626203  \\ 
lda &   0.9498792 \\ 
rf &   0.9709806 \\ 
tree &   0.8468115\\ 
   \hline
\end{tabular}
	\label{tabla:AccuracyCV}
	\caption{Accuracy usando validación cruzada}
\end{table}
\end{frame}


\begin{frame}
\end{frame}
\end{document}
